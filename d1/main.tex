\documentclass[letterpaper]{report}
\usepackage{geometry}
\usepackage{graphicx}
\usepackage{xspace}
\usepackage[dvipsnames]{xcolor}
\fboxsep0pt

\usepackage{caption}
\usepackage{glossaries}

\usepackage{hyperref}
\hypersetup{
  colorlinks=true,
  linkcolor=blue,
  filecolor=magenta,     
  urlcolor=cyan,
  pdftitle={. . .},
  pdfpagemode=FullScreen,
}
\urlstyle{same}

\renewcommand{\labelenumii}{\arabic{enumi}.\arabic{enumii}}
\usepackage[inline]{enumitem}
\newcommand{\METRICSTICS}{\texttt{METRICSTICS}\xspace}

% Cover Page - Start

\title{Software Measurement (SOEN 6611)\\[.5em]
METRICSTICS\\[.5em]
Deliverable - 1\\[.5em]
Team - 'R'\\[.5em]}
\author{Shashank Verma, Revanth Velagandula, Sri Neha Velagapudi, \\[.5em]
Manasa Yalakala, Manaswini Yarlagadda, Wenxue Zhao\\[.5em]}

\date{October 2, 2023\\[.5em]}

% Cover Page - End


\begin{document}
\maketitle

\tableofcontents

% \listoffigures
% \listoftables



% delete this section if no figures
\listoffigures\addcontentsline{toc}{chapter}{List of Figures}
% This will automatically be populated if you included figures in your report.


%%%%%
\chapter{Background Information}

The purpose of descriptive statistics is to quantitatively describe a collection of data by measures of central tendency, measures of frequency, and measures of variability. \\
\\Let x be a random variable that can take values from a finite data set x1, x2, x3, ..., xn, with each value having the same probability. \\
\\The minimum, m, is the smallest of the values in the given data set. (m need not be unique.)\\
\\The maximum, M, is the largest of the values in the given data set. (M need not be unique.)\\
\\The mode, o, is the value that appears most frequently in the given data set. (o need not be unique.)\\
\\The median, d, is the middle number if n is odd, and is the arithmetic mean of the two middle numbers if n is even.\\
%
% Arithmetic Mean
%
\\The arithmetic mean, $\mu$, is given by
\[ \mu = \frac{1}{n} \sum_{i=1}^{n} x_i \]\\
%
% Mean Absolute Deviation
%
\\The mean absolute deviation, MAD, is given by
\[ \text{MAD} = \frac{1}{n} \sum_{i=1}^{n} |x_i - \mu| \]\\
%
% Standard Deviation
%
The standard deviation, $\sigma$, is given by
\[ \sigma = \sqrt{\frac{1}{n}\sum_{i=1}^{n}(x_i - \mu)^2} \]\\
\\Let there be a system, called METRICSTICS (a portmanteau of METRICS and
STATISTICS), for finding m, M, o, d, $\mu$, MAD, and $\sigma$. The system must take as input a random number of data values and output its descriptive statistics. 







\chapter{Problem 1}
\section{Introduction}

\section{. . .}

\subsection{. . .}

\section{...}

\begin{thebibliography}{. . .}

\bibitem{. . .}
\textit{. . .}

\end{thebibliography}
\end{document}